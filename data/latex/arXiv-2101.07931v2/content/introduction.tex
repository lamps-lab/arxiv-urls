\section{Introduction}
Without an effective curative or preventative measure, the unprecedented coronavirus disease 2019 (COVID-19) pandemic has led to a significant amount of human deaths (1,900,000 at the time of publication (\cite{webtrack})). However, now with the advent of vaccines, we face the challenges of strategic, equitable and privacy preserved ways for last-mile vaccine distribution (\cite{bae2020challenges, vaccinetrack}).   

First, the vaccine recipients must be dynamically prioritized to ensure an equitable reach, especially as multiple vaccines with different protocols are approved in various areas. In addition, once a citizen’s first dose is administered, they must follow through with their second dose as well. Also, a communication plan must also be put in place to combat inevitable rumours, misinformation, and conspiracy theories aiming to disrupt citizen engagement in the vaccination process (~\cite{morales2021covid19, wp_article}). It must also address the mistrust of vaccines in society (\cite{Palamenghi2020}), especially within previously marginalized minority populations (\cite{JHU}). This is why we must take a user-centric approach that preserves trust — vaccines are meaningless if citizens aren’t willing to take them (\cite{PMC}). Lastly, the health outcomes (effectiveness, safety, long-term effects, etc) of the vaccines must be effectively monitored in a privacy-preserving way (\cite{hbr}).

In today’s society, multiple technological systems are being utilized by the Center for Disease Control (CDC) to combat these challenges (~\cite{cdc1,cdc2,cdc3}). For example, the Vaccine Administration Management System (VAMS) streamlines the vaccine distribution process for jurisdictions, employers, and healthcare providers. In addition, it’s an effective user-centric system as it allows for vaccine recipients to schedule appointments, receive records of their visit, and receive reminders for a second dose (\cite{VAMS}). The Immunization Information Systems (IIS) are a group of privacy preserving database systems that track all vaccinations within various areas (\cite{IIS}). Lastly, the Vaccine Adverse Event Reporting System (VAERS) is the prominent system for the monitoring of health outcomes (~\cite{VAERS,vaers_article}).

In our previous work, we detail the MIT SafePaths app-based protocol for vaccine distribution. In this paper, we introduce a separate user-centric card protocol that uses printed codes as a supplement to traditional paper based vaccination cards. 