\section{Threats to validity}\label{sec:threats-to-validity}
Our study suffers from the following threats to validity.

\textbf{Subject selection bias.} The performance of the model depends on the data. Since stack traces for the bug triage task are being used for the first time, there is no dataset available for this task. We collected a dataset from the products of a large software company and evaluated the proposed approach on them. However, applying the model to other dataset may lead to different results. For instance, workflows in open-source projects could be more volatile and unstable. The results for such datasets can be noticeably lower.

\textbf{Limited scope of application.} Our solution is applicable for software systems that report stack traces when a bug happens, which might be not be typical for some projects and companies. However, we believe this practice to be common enough for our approach to be helpful in practice. 
Secondly, deep learning models are over-parameterized. A modern neural network contains thousands or millions of parameters. A sufficient amount of data is required to train a neural network. We use 11,139 different stack traces in our private dataset and regularization techniques to prevent overfitting. However, in cases when this amount of data is not available, the results may differ. We hope our research will encourage other researchers and practitioners to invest time and effort into collecting a larger dataset of such kind.

\textbf{Programming language bias.} Our datasets consist of stack traces that were obtained from the JVM languages. Therefore, the results of our models for other languages may differ. Firstly, stack trace characteristics change from one language to another. The performance of the model depends on the average length of the stack trace, as well as the variety of methods and files used. Secondly, an essential component of our approach is the use of features from annotations. The characteristics of these files also strongly affect the model performance. The distribution of developers for each file can vary between teams, companies, and maybe even programming languages. Future research is needed to assess how much all of this affects the resulting model. 

While these threats to validity are important to note, we believe that they do not invalidate the overal results of our study and its practical usefulness.